\section{Related work}
Several works have been proposed in the literature to the plant disease identification. The classical approach given by the expertise support directly on the field has offered diverse solutions such as: hyperspectral proximal sensing techniques to evaluate plant stress to environmental conditions \cite{ref14}, optical technologies like thermal and fluorescence imaging methods for estimating plant stress produced mainly by increased gases, radiation, water status, and insect attack, among others \cite{ref15}, chemical elements were applied to leaves in order to estimate their defense capabilities against pathogens \cite{ref16}. The previous methods show outstanding performance, but they do not provide yet a scalable and cost-effective solution \cite{ref17}.
\\\indent
After analysis of their work and investigation presented by the authors of \cite{ref18, ref19}, it was decided to use the image processing approach among other approaches, for instance, double-stranded ribonucleic acid (RNA) analysis, nucleic acid probes, and microscopy. Several handcrafted feature-based methods have been widely applied specifically for image classification. Some of the best-known handcrafted feature methods are: the Histogram of Oriented Gradients (HOG) \cite{ref20} and Scale-Invariant Feature Transform (SIFT) \cite{ref21}; YcbCr, HSI, and CIELB colour models \cite{ref22} effective against noise from different sources; extracting shape feature method to determine leaf and lesioning area \cite{ref23}; extracting texture feature such as inertia homogeneity, and correlation. Combination of all these features provides a robust feature set for image improvement and better classification. In \cite{ref28}, the authors have presented a survey of well-known conventional methods of feature extraction. These methods are usually combined with classifiers (e.g., Support Vector Machines \cite{ref24}, Neural Networks \cite{ref25}, Adaptive Boosting \cite{ref26}, K-Nearest Neighbors \cite{ref27}, ensemble methods \cite{ref28}). 
\\\indent
However, Deep Learning has allowed researchers to consider and design systems as a unified process \cite{ref29}. In particular, Convolutional Neural Networks (CNNs), first introduced in \cite{ref30}, showed, in fact, how to bridge feature extraction to classification in image recognition task by means of the LeNet architecture. For the first time, the authors of \cite{ref31} applied the principle of CNN to plant diseases recognition in different crops using LeNet and image processing to recognize two leaf diseases out of healthy ones. 
\\\indent
Other works that use deep convolutional neural networks for disease recognition have been proposed, showing good performance on different crops.